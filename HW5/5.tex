\documentclass[addpoints]{exam}
\usepackage{amsmath,amsthm,amssymb,url}

\usepackage{algorithm}
\usepackage{algorithmic}
\usepackage{graphicx}
\usepackage{float}
\usepackage[pdftex]{hyperref}


\newtheorem{lemma}{Lemma}[section]
\newcommand{\var}{\text{Var}}
\title{CS 6150: HW4}
\date{Due Date: }
\begin{document}
\maketitle
\begin{center}
\fbox{\fbox{\parbox{5.5in}{\centering
This assignment has \numquestions\ questions, for a total of \numpoints\
points.
Unless otherwise specified, complete and reasoned arguments will be
expected for all answers. 
}}}
\end{center}

\qformat{Question \thequestion: \thequestiontitle\dotfill \textbf{[\totalpoints]}}
\pointname{}
\bonuspointname{}
\pointformat{[\bfseries\thepoints]}

\printanswers

\begin{center}
  \gradetable
\end{center}
\newpage



\begin{questions}

\titledquestion{Integer programs}
Write down integer programs for the following problems.
\begin{parts}
  \part[10] Let $U$ be a set, and let $\mathcal{C} = \{S_1, \ldots, S_n\}$ be a
  collection of subsets of $U$. Each set $S$ has a weight $w_S$. Find a subcollection $\mathcal{C}' \subset
\mathcal{C}$ of minimum total weight ($w(\mathcal{C}') = \sum_{S \in
  \mathcal{C}'} w_S$) such that the sets in $\mathcal{C}'$
\emph{cover} $U$: i.e
\[ \cup_{S \in \mathcal{C}'} S = U \]
\part[10] Let $U$ be a universe, and let $\mathcal{C} = \{S_1, \ldots, S_n\}$ be a
  collection of subsets of $U$. Each element $u \in U$ has a weight $w_u$. Find
  a subset $H \subset U$ of minimum total weight ($w(H) = \sum_{u \in H} w_u$) such that each set
  in $\mathcal{C}$ is \emph{hit} by $H$: i.e
\[ \forall S \in \mathcal{C}, H \cap S \ne \emptyset \]
\end{parts}

\titledquestion{LP duals}
\begin{parts}
  \part[10] Consider the linear program 
  \begin{align*}
    \max\qquad & 5x + 3y - 2z \\
    \text{such that}\quad     & 3x - 2y \le 6 \\
         & 4y + 2z \le 7 \\
         & -3x + 2z \le 3
  \end{align*}
Write down the dual of this LP. 
\part[10] Write down the dual of the linear program obtained by relaxing the
integer program from Question 1(b) above. 
\end{parts}

\titledquestion{Max cardinality matching}
\begin{parts}
  \part[10] Write down a linear program for computing a maximum cardinality
  matching in a bipartite graph. Your linear program will have one variable for
  each edge. 
\part[10] Write down the dual of this LP. What well known problem does it
capture ? 
\end{parts}


\titledquestion{Generalized Duals}[20]

We've seen that any linear program can be written in the canonical form 
\begin{align*}
  \max \qquad & c^\top x \\
  \text{such that} \quad & Ax \le b \\
  & x \ge 0
\end{align*}

which gives rise to the corresponding dual
\begin{align*}
  \min \qquad & y^\top b \\
  \text{such that} \quad & y^\top A \ge c \\
  & y \ge 0
\end{align*}
It turns out that first transforming a general linear program with equality and
$\ge$ constraints into canonical form, and then writing the dual, can be a
little inconvenient, and that it's easier to write the dual directly. 

But what would this dual look like ? Let's take a general linear program that
looks like this: 

\begin{align*}
\max \qquad & ax + by + cz \\
\text{such that} \quad & Ax + By + Cz \le d \\
& Dx + Ey + Fz = e \\
& Gx + Hy + Iz \ge f \\
& x \ge 0, z \le 0
\end{align*}

Note that $x, y, z$ are \emph{vectors} and $y$ is unconstrained (i.e the
coordinates of $y$ could be more or less than zero). 

Write down the dual of this linear program. You will do this by first
transforming this into the canonical setting, writing the canonical dual, and
then rewriting the dual in simplified form. It will help to remember that if $a$
and $b$ are two variables that are both greater than zero, then $a-b$ represents
a variable that could be either more or less than zero. 

Do you notice any pattern in the relation between primal constraint and dual
variables (and vice versa) ? 
\titledquestion{Best fit line}[20]
\emph{this is question 7(b) from the Erickson notes on LPs}

You are given $n$ points $(x_i, y_i)$ in the plane, and you wish to find a line
of \emph{best fit}. But instead of the standard squared error norm, you will be
using the $\ell_1$ error: namely, for any given line $y = ax + b$, the error is
given by 
\[ \epsilon_1(a, b) = \sum_{i=1}^n |y_i - (ax_i + b)|. \]
Write down a linear program to find a line that minimizes $\epsilon_1$. 
\end{questions}

\end{document}

%%% Local Variables:
%%% mode: latex
%%% TeX-master: t
%%% End:
